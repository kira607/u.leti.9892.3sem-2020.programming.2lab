\section*{Описание пользовательских функций}

\subsection*{GenElement}

\begin{lstlisting}[label={lst:GenElement}]
	int GenElement(GenType gen_type, int width, int i, int j);
\end{lstlisting}

Функция для генерации элементов матрицы.

Принимает на вход тип генерации, ширину матрицы,
строку и столбец элемента.

Возвращает число -- элемент матрицы.

Алгоритм работы функции:

В зависимости от типа генерации
\begin{itemize}
	\item Если GenType::Zero вернуть 0
	\item Если GenType::One вернуть 1
	\item Если GenType::Random вернуть случайное число от 0 до 9
	\item Если GenType::Fill0 вернуть $ i*width+j $
	\item Если GenType::Fill1 вернуть $ i*width+j+1 $
	\item Если GenType::Input вернуть InputElement(i, j)
	\item В остальных случаях вернуть -1
\end{itemize}

\subsection*{Input}

\begin{lstlisting}[label={lst:Input}]
	int Input(const std::string& message = "Input: ", int l = _min_int, int r = _max_int);
\end{lstlisting}

Функция для ввода числа.
Содержит в себе логику по обработке исключений и
ограничения диапазона допустимых чисел.

Принимает на вход сообщение, приглашающее к вводу,
левую и правую границы диапазона ввода.

Возвращает введённое число

Алгоритм работы функции:

\begin{itemize}
	\item int element;
	\item bool input = true;
	\item Пока input == true
	\subitem Вывести сообщение
	\subitem Ввести элемент
	\subitem Если ошибка ввода или введённый элемент не входит в допустимый диапазон
	\subsubitem Выводим "Wrong input!"
	\subitem Иначе
	\subsubitem input = false;
	\subitem Чистим поток cin от мусора
	\item Возвращаем element
\end{itemize}

\subsection*{InputElement}

\begin{lstlisting}[label={lst:InputElement}]
	int InputElement(int i, int j);
\end{lstlisting}

Функция для ввода элемента матрицы.
Является обёрткой над функцией Input().

Принимает на вход строку и столбец вводимого элемента.

Возвращает вводимый пользователем элемент

Алгоритм работы функции:

\begin{itemize}
	\item Создаём сообщение "Input element [i][j]: ", подставляя вместо i и j входные параметры
	\item Возвращаем Input()
\end{itemize}

\subsection*{InputDimensions}

\begin{lstlisting}[label={lst:InputDimensions}]
	Matrix2 InputDimensions();
\end{lstlisting}

Функция для ввода размеров матрицы.
Является обёрткой над функцией Input().

Не принимает на вход параметров

Возвращает матрицу с определёнными размерами.

Алгоритм работы функции:

\begin{itemize}
	\item Matrix2 result{};
	\item Записываем в ширину результат Input("Width: ", 1);
	\item Записываем в высоту результат Input("Height: ", 1);
	\item Возвращаем результирующую матрицу
\end{itemize}

\subsection*{CreateMatrix}

\begin{lstlisting}[label={lst:CreateMatrix}]
	Matrix2 CreateMatrix(int w, int h, GenType gen_type = GenType::Zero);
\end{lstlisting}

Функция для создания матрицы.

Принимает на вход ширину, высоту и тип генерации матрицы.

Возвращает матрицу.

Алгоритм работы функции:

\begin{itemize}
	\item Создаём результирующую матрицу
	\item Записываем в ширину параметр ширины
	\item Записываем в высоту параметр высоты
	\item Выделяем память под строки (размером $ height $)
	\item Цикл по i от 0 до $ height $
	\begin{itemize}
		\item Выделяем память под элементы строки (размером $ width $)
		\item Цикл по j от 0 до $ width $
		\begin{itemize}
			  \item int new\_element = GenElement(gen\_type, result.width, i, j);
			  \item result.matrix[i][j] = new\_element;
		\end{itemize}
	\end{itemize}
	\item Возвращаем результирующую матрицу
\end{itemize}

\subsection*{LoadMatrix}

\begin{lstlisting}[label={lst:LoadMatrix}]
	Matrix2 LoadMatrix(const std::string& file_name);
\end{lstlisting}

Функция для загрузки матрицы из файла.

Принимает на вход имя файла (путь к файлу).

Возвращает матрицу.

Алгоритм работы функции:

\begin{itemize}
	\item Создаём результирующую матрицу
	\item Пытаемся открыть файл
	\item Если файл не открыт, возвращаем пустую матрицу
	\item Считываем из файла параметры ширины и высоты
	\item Выделяем память под строки (размером $ height $)
	\item Цикл по i от 0 до $ height $
	\begin{itemize}
		\item Выделяем память под элементы строки (размером $ width $)
		\item Цикл по j от 0 до $ width $
		\begin{itemize}
			\item fin >> new\_element;
			\item result.matrix[i][j] = new\_element;
		\end{itemize}
	\end{itemize}
	\item Возвращаем результирующую матрицу
\end{itemize}

\subsection*{String}

\begin{lstlisting}[label={lst:String}]
	std::string String(const Matrix2 &m);
\end{lstlisting}

Функция для получения форматированного строкового представления матрицы.

Принимает на вход константную ссылку на матрицу

Возвращает форматированное строковое представление матрицы

Алгоритм работы функции:

\begin{itemize}
	\item Создаём строковый поток
	\item Цикл по i от 0 до $ height $ матрицы
	\begin{itemize}
		\item Цикл по j от 0 до $ width $ матрицы
		\begin{itemize}
			\item Записываем в поток смещённый вправо на 5 символов элемент матрицы\\
			ss << std::right << std::setw(5) << m.matrix[i][j] << '' '' ;
		\end{itemize}
		\item Записываем в поток символ конца строки
	\end{itemize}
	\item Возвращаем строковое представление потока
\end{itemize}

\subsection*{Print}

\begin{lstlisting}[label={lst:Print}]
	void Print(const Matrix2 &m);
\end{lstlisting}

Функция для печати матрицы в стандартный поток вывода.

Принимает на вход константную ссылку на матрицу

Не возвращает ничего.

Алгоритм работы функции:

\begin{itemize}
	\item Вывести в stdout результат String(m);
\end{itemize}

\subsection*{Sum}

\begin{lstlisting}[label={lst:Sum}]
	Matrix2 Sum(Matrix2 a, Matrix2 b);
\end{lstlisting}

Функция для суммирования двух матриц.

Принимает на вход две константные ссылки на матрицы.

Возвращает матрицу -- сумму двух переданных матриц.

Алгоритм работы функции:

\begin{itemize}
	\item Если размеры матриц не совпадают, выкидываем исключение.
	\item Создаём результирующую матрицу
	\item Цикл по i от 0 до $ height $
	\begin{itemize}
		\item Цикл по j от 0 до $ width $
		\begin{itemize}
			\item result.matrix[i][j] = a.matrix[i][j] + b.matrix[i][j];
		\end{itemize}
	\end{itemize}
	\item Возвращаем результирующую матрицу
\end{itemize}

\subsection*{Delete}

\begin{lstlisting}[label={lst:Delete}]
	void Delete(Matrix2 m);
\end{lstlisting}

Функция для удаления матрицы из динамической памяти.

Принимает на вход ссылку на матрицу.

Не возвращает ничего.

Алгоритм работы функции:

\begin{itemize}
	\item Цикл по i от 0 до $ height $
	\begin{itemize}
		\item Освобождаем память $ i $ -ой строки.
	\end{itemize}
	\item free(m.matrix);
\end{itemize}

\subsection*{PrintMainMenu}

\begin{lstlisting}[label={lst:PrintMainMenu}]
	void PrintMainMenu();
\end{lstlisting}

Функция для вывода пользовательского меню.

Не принимает на вход параметров

Не возвращает ничего.

Алгоритм работы функции:

\begin{itemize}
	\item Вывести меню
\end{itemize}

\subsection*{ChooseGenType}

\begin{lstlisting}[label={lst:ChooseGenType}]
	GenType ChooseGenType();
\end{lstlisting}

Функция для пользовательского выбора типа генерации матрицы.

Не принимает на вход параметров.

Возвращает выбранный пользователем тип генерации.

Алгоритм работы функции:

\begin{itemize}
	\item Выводим меню
	\item Считываем пользовательский ввод (в диапазоне 1--7)\\
	int option = Input(''Input: '', 1, 7);
	\item Возвращаем считанное число, приведённое к типу GenType
\end{itemize}

\subsection*{TryLoad}

\begin{lstlisting}[label={lst:TryLoad}]
	bool TryLoad(Matrix2 &m);
\end{lstlisting}

Функция для обработки загрузки матрицы.
Включает в себя обработку исключений.

Принимает на вход ссылку на матрицу.

Возвращает bool -- успешность загрузки.

Алгоритм работы функции:

\begin{itemize}
	\item Считываем имя файла
	\item Matrix2 lm = LoadMatrix(file\_name);
	\item Если размеры матриц не совпадают или матрица пустая
	\begin{itemize}
		\item Выводим сообщение об ошибке
		\item return false;
	\end{itemize}
	\item m = lm;
	\item return true;
\end{itemize}

\subsection*{Menu}

\begin{lstlisting}[label={lst:Menu}]
	void Menu(Matrix2 &m);
\end{lstlisting}

Функция-меню для взаимодействия с пользователем и создания матрицы.

Принимает на вход ссылку на матрицу.

Не возвращает ничего.

Алгоритм работы функции:

\begin{itemize}
	\item bool continue\_ = true;
	\item Пока continue:
	\begin{itemize}
		  \item Считываем тип генерации
		  \item Если тип генерации НЕ файл
		  \subitem Создаём матрицу на основе типа генерации
		  \subitem continue\_ = false;
		  \item Иначе
		  \subitem continue\_ = !TryLoad(m);
	\end{itemize}
	\item Выводим полученную матрицу на экран Print(m);
\end{itemize}